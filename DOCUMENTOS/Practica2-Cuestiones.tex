% !ht TEX root = Practica2-Cuestiones.tex

\documentclass[12pt,a4paper]{report}

% Paquetes útiles
\usepackage[utf8]{inputenc}    % Acentos
\usepackage[T1]{fontenc}
\usepackage[spanish]{babel}    % Idioma español
\usepackage{graphicx}          % Imágenes
\usepackage{geometry}          % Márgenes
\usepackage{setspace}          % Interlineado
\usepackage{titlesec}          % Controlar formato de capítulos
\usepackage{hyperref}
\usepackage{pgfplots}
\pgfplotsset{compat=1.18}
\usepackage{caption}
\usepackage{float}

%\usepackage{arial}
% Formato de capítulos: menos espacio arriba
\titleformat{\chapter}[block]
  {\normalfont\huge\bfseries}
  {\thechapter.}{1em}{}
\titlespacing*{\chapter}{0pt}{-40pt}{20pt}

% Configuración de márgenes
\geometry{left=2cm,right=2.5cm,top=3cm,bottom=3cm}

\begin{document}

% ==============================
% PORTADA
% ==============================
\begin{titlepage}
    \centering
    {\scshape\LARGE Universidad de Murcia\par}
    \vspace{1cm}
    {\scshape\Large Facultad de Ingeniería\par}
    \vspace{3cm}
    {\Huge\bfseries Práctica 2 - Cuestiones\par}
    \vspace{1cm}
    {\Large Sistemas Inteligentes\par}
    \vfill
    \begin{flushleft}
        \textbf{Autor:}\\
            Andrey Santiago Morales Vicuña \\
        Grupo: 2.3 \\
        Correo: asantiago.morales@um.es
    \end{flushleft}
    \vspace{1cm}

    \vfill
    {\large \today\par}
\end{titlepage}

\tableofcontents
\newpage

\chapter{Ejercicios pedidos}

\begin{enumerate}

    \item \textbf{Explicación breve de los tres elementos de los que consta un Sistema basado en reglas (SBR).}

        Un SBR consta de:
        \begin{itemize}
        \item Una \textbf{Base de Hechos (BH)} : Es la representación del estado actual de resolución del problema, 
        contiene hechos verdaderos, tanto datos de entrada como conclusiones inferidas, metas o subproblemas.
        Los datos de entrada pueden ser introducidos por el usuario, de sistemas externos, iniciales o introducidos durante el proceso. 
        Se diferencia de la BC porque es dinámico, se modifica su contenido durante el proceso, tras verificar reglas.

        \item Una \textbf{Base de Conocimientos (BC)} : Es estático, agrupa el conjunto de reglas que codifican todo el conocimiento 
        experto sobre cómo resolver el problema. Una
        regla consta de dos partes:
            \begin{itemize}
                \item Parte izquierda, denominada condición o antecedente.
                \item Parte derecha, denominada acción o consecuente.
            \end{itemize}
        Podemos definir una regla como un par \textit{condición-acción}.
        
        \item Un \textbf{Mecanismo de Inferencias (MI)} : Selecciona las reglas que se pueden aplicar y las ejecuta, con el objetivo
        de obtener alguna conclusión. Son las reglas que podoemos aplicar en cada momento para obtener nuevos conocimientos. Es un mecanismo
        algorítmico para obtener conclusiones aplicando la BC a loshechos conocidos almacenados en la BH. Las conclusiones se introducen, a su
         vez, en la BH. Entrada BH y BC, Salida BH' (modificada).

        \end{itemize}

    \item \textbf{Explicación breve de cómo se representa el conocimiento incierto mediante Factores de Certeza.}
    
        Para tratar de representar hechos cuya fiabilidad o precision es limitada y de que los conocimientos que tenemos 
        no con absolutamente ciertos, utilizamos los Factores de Certeza, que básicamente son la creencia que tenemos sobre la hipótesis en función
        de las evidencias o datos conocidos. Estos valores son propocionados por los expertos.

    \item \textbf{¿Qué es lo que mide un factor de certeza FC asociado a un hecho A?}
        
        MC mide hasta que punto la evidencia apoya la hipótesis.
        MI mide hasta que punto la evicencia apoya la negación de la hipótesis. Si una tiene el valor máximo, la otra es 0 siempre. Por lo que 
        el FC está entre -1 y 1.
        El factor de certeza se define a partir de la medida de la creencia y la de incertidumbre como: 
        FC = MC - MI 

        Por tanto si FC(A) vale 0,5 podemos decir que creemos 0,5 en que es verdad, y si es -0.5 quiere decir que estoy a la mitad de dejar de creer que 
        no es verdad para pasar a creer que es verdad (pero sigo creyendo no es verdad) 

    \newpage

    \item \textbf{¿Qué es lo que dirías sobre “culpable” con la siguiente información?}
    \begin{enumerate}
        \renewcommand{\labelenumi}{\alph{enumi})}
        \item Hemos obtenido en un proceso de inferencia el hecho “culpable” con FC=0.9 
            Estoy casi seguro que es culpable
        \item Hemos obtenido en un proceso de inferencia el hecho “culpable” con FC=0
            No tengo información para decidir algo.
        \item Hemos obtenido en un proceso de inferencia el hecho “culpable” con FC=-0.1
            Estoy a punto de creer que es culpable. No creo casi nada que no sea culpable.
    \end{enumerate}
    \item \textbf{¿Para qué se necesita utilizar el CASO 2 durante el proceso de inferencia?}
        Porque puedo adquirir conocimientos de varios lados y tengo que contrastarlos para un veredicto final.

    \item \textbf{Disponemos de una BC compuesta de un conjunto de reglas Ri las cuales utilizan 4 hechos (A, 
            B, C, D). Si para un proceso de inferencia nos proporcionan FCs de los hechos A, C y D, ¿Qué 
            debemos hacer con el hecho B? ¿Por qué? Si lo utilizamos, ¿qué FC se le asignaría? ¿Por qué? }

        Usarlo normal pero con el valor que le hemos asignado, le asignamos 0 porque con eso representamos que no tenemos conocimiento 
        del hecho, no podemos darle otro valor porque de eso se encargan los expertos porque podríamos estar equivocados e influir 
        en el veredicto final.

\end{enumerate}

\end{document}


