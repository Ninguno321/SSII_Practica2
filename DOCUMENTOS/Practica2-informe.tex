% !ht TEX root = Practica2-informe.tex

\documentclass[12pt,a4paper]{report}

% Paquetes útiles
\usepackage[utf8]{inputenc}    % Acentos
\usepackage[T1]{fontenc}
\usepackage[spanish]{babel}    % Idioma español
\usepackage{graphicx}          % Imágenes
\usepackage{geometry}          % Márgenes
\usepackage{setspace}          % Interlineado
\usepackage{titlesec}          % Controlar formato de capítulos
\usepackage{hyperref}
\usepackage{pgfplots}
\pgfplotsset{compat=1.18}
\usepackage{caption}
\usepackage{float}


% Formato de capítulos: menos espacio arriba
\titleformat{\chapter}[block]
  {\normalfont\huge\bfseries}
  {\thechapter.}{1em}{}
\titlespacing*{\chapter}{0pt}{-40pt}{20pt}

% Configuración de márgenes
\geometry{left=2cm,right=2.5cm,top=3cm,bottom=3cm}

\begin{document}

% ==============================
% PORTADA
% ==============================
\begin{titlepage}
    \centering
    {\scshape\LARGE Universidad de Murcia\par}
    \vspace{1cm}
    {\scshape\Large Facultad de Ingeniería\par}
    \vspace{3cm}
    {\Huge\bfseries Práctica 2 - Informe\par}
    \vspace{1cm}
    {\Large Sistemas Inteligentes\par}
    \vfill
    \begin{flushleft}
        \textbf{Autor:}\\
            Andrey Santiago Morales Vicuña \\
        Grupo: 2.3 \\
        Correo: asantiago.morales@um.es
    \end{flushleft}
    \vspace{1cm}

    \vfill
    {\large \today\par}
\end{titlepage}

\tableofcontents
\newpage

\chapter{Ejercicios pedidos}

\begin{enumerate}

    \item \textbf{Explicación breve de los tres elementos de los que consta un Sistema basado en reglas (SBR).}

    \item \textbf{Explicación breve de cómo se representa el conocimiento incierto mediante Factores de Certeza.}

    \item \textbf{¿Qué es lo que mide un factor de certeza FC asociado a un hecho A?}

    \item \textbf{¿Qué es lo que dirías sobre “culpable” con la siguiente información?}
    \begin{enumerate}
        \renewcommand{\labelenumi}{\alph{enumi})}
        \item Hemos obtenido en un proceso de inferencia el hecho “culpable” con FC=0.9 
        \item Hemos obtenido en un proceso de inferencia el hecho “culpable” con FC=0
        \item Hemos obtenido en un proceso de inferencia el hecho “culpable” con FC=-0.1 
    \end{enumerate}
    \item \textbf{¿Para qué se necesita utilizar el CASO 2 durante el proceso de inferencia?}

    \item \textbf{Disponemos de una BC compuesta de un conjunto de reglas Ri las cuales utilizan 4 hechos (A, 
            B, C, D). Si para un proceso de inferencia nos proporcionan FCs de los hechos A, C y D, ¿Qué 
            debemos hacer con el hecho B? ¿Por qué? Si lo utilizamos, ¿qué FC se le asignaría? ¿Por qué? }

            



\end{enumerate}

\end{document}


